\documentclass[12pt,a4paper,titlepage]{article}
\usepackage[utf8]{inputenc}
\usepackage[spanish]{babel}
\usepackage[T1]{fontenc}
\usepackage[pdftex]{color,graphicx}
\usepackage{listings}
\usepackage{inconsolata}

\definecolor{gray}{RGB}{127,127,127}
\newcommand{\HRule}{\rule{\linewidth}{0.5mm}}

\lstset{
language=Java,                          	% Code langugage
basicstyle=\ttfamily,                   	% Code font
numbers=left,                           	% Line nums position
numberstyle=\tiny,                      	% Line-numbers fonts
stepnumber=1,                           	% Step between two line-numbers
numbersep=5pt,                          	% How far are line-numbers from code
frame=none,                             	% A frame around the code
tabsize=4,                              	% Default tab size
captionpos=b,                           	% Caption-position = bottom
breaklines=true,                        	% Automatic line breaking?
breakatwhitespace=false,                	% Automatic breaks only at whitespace?
showspaces=false,                       	% Dont make spaces visible
showtabs=false,                         	% Dont make tabls visible
linewidth=\textwidth,                   	% Defines the base line width
commentstyle=\color{gray}
}

\author{
	Daniel Garabato Míguez
	\and Vanesa López Beade
	\and Daniel Valcarce Silva
}
\title{Memoria Práctica LN}
\date{Curso 2012-2013}

\begin{document}

% Título
\begin{titlepage}
\begin{center}
\includegraphics[width=10cm]{logo_udc}\\
\vspace{1cm}
\textsc{\Large Lenguajes Naturales}\\[0.5cm]
\textsc{\Large Curso 2012/2013}\\[0.5cm]

\HRule \\[0.4cm]
{ \huge \bfseries My Little Trivial Player}\\[0.4cm]
{ \Large \bfseries Memoria de la Práctica}\\[0cm]

\HRule \\[0cm]
\end{center}

\vfill
\emph{Autores:}
\vspace{0.5cm}
\\
\vspace{0.1cm}
Garabato Míguez, Daniel \texttt{<daniel.garabato@udc.es>}\\
\vspace{0.1cm}
López Beade, Vanesa \texttt{<vanesa.lopezb@udc.es>}\\
\vspace{0.1cm}
Valcarce Silva, Daniel \texttt{<daniel.valcarce@udc.es>} (Portavoz)\\

\end{titlepage}


% Índice
\tableofcontents
\newpage

%Cuerpo
\section{Arquitectura del sistema}



\newpage
\section{Herramientas empleadas}

\subsection{Toolkits}
NLTK

\subsection{Sistema de control de versiones}
Git y Bitbucket


\newpage
\section{Manual de instalación y uso}

\subsection{Instalación del sistema}


\subsection{Uso del sistema}



\newpage
\section{Diario de trabajo}
A continuación se muestra un resumen del trabajo realizado a lo largo de la elaboración de la presente práctica. Las entradas se ordenan por su fecha cronológica y describen brevemente las decisiones y las acciones tomadas.


\subsubsection*{2012/10/26}
En la primera reunión del grupo, hemos discutido sobre los primeros pasos a la hora de enfrentar la práctica. Tras un análisis de los diferentes \emph{toolkits} que se nos presentan en el enunciado de la práctica, nos decidimos a usar NLTK por dos razones principales. El primer motivo es la gran cantidad de módulos que posee y su amplia documentación. En segundo lugar, porque Python nos parece un lenguaje muy cómodo para el desarrollo del proyecto.

Profundizando más en el desarrollo del trabajo, decidimos usar Git como sistema de control de versiones y apoyarnos en un repositorio privado de Bitbucket. Esto nos permitirá tener nuestro código bien organizado y documentado así como proporcionarnos un respaldo de los datos.

Por último, acordamos documentarnos más sobre el uso de Python en el procesamiento de lenguaje natural en general y con NLTK en particular. Para ello recurriremos a la bibliografía recomendada por los creadores del toolkit \cite{nltk}. También optamos por estudiar las APIs de Google y de Bing para realizar consultas.

\subsubsection*{2012/10/29}


\newpage
\begin{thebibliography}{99}
\bibitem{nltk}
S. Bird, E. Klein, E. Loper: \emph{Natural Language Processing with Python --- Analyzing Text with the Natural Language Toolkit}, O'Reilly Media (2009)
\end{thebibliography}


\end{document}
